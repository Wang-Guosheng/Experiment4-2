\par 一般实验条件下可得到的磁感应强度有限,磁共振所涉及的共振频率处于射频与微波频段。对磁化强度较小的材料(比如核磁共振)或者当较弱的磁场(比如光泵磁共振),常用射频电磁波做磁共振的激励信号即可;但在铁磁和电子顺磁共振中,磁矩较大,磁效应比较显著,常采用频率更高的微波观测其磁共振效应。因为微波的能量量子大约是10$^{-6\sim -3}$eV,而许多原子和分子发射和吸收的电磁波的能量正好处在这个波段,所以微波与特定物质相互作用时可以发生铁磁共振。
\par 本实验首先说明微波体效应振荡器的工作原理、微波传输的基本原理、方法和元件,测试微波体效应二极管和传输式谐振腔的工作特性, 进而通过测量谐振腔中的微波场随磁场的变化观测铁磁共振现象,得出铁氧体样品的相关参数。