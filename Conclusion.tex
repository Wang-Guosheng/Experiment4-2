\par 实验测量了微波体效应管的工作特性,并用它产生的微波激发了铁氧体样品的铁磁共振,进而测出了样品的相关参数。
\par 实验中首先测出了微波体效应管的电流-电压关系和频率-电压关系。发现体效应管的电流随电压的增大先增大后减小再增大或基本稳定,中间有一段负阻区,微分电阻小于零;分析表明负阻区是由高场畴的形成过程引起的,正是利用这一负阻效应,我们得以通过砷化镓晶体管和提供带适当直流偏压的交变电压的外电路的相互作用产生振荡电流,发射微波。测量还表明体效应管的输出频率和电压在10-13V内呈负相关,推测是工作电压对高场畴渡越过程的影响造成的。我们随后还标定了该体效应管的频率-输出功率关系。
\par 此后我们通过测量谐振腔的谐振曲线求出了该体效应管的有载品质因数$Q=3410$。而后用示波器分别观察了单晶样品和多晶样品的共振曲线,发现后者的共振线宽较大,弛豫时间较长,这说明铁氧体多晶是由许多共振频率不完全相同的单晶组成的。最后,我们先用逐点法具体测定了多晶样品的铁磁共振曲线,又用高斯计标定了电磁铁电流与磁感应强度的关系,进而算得了多晶样品的共振线宽$\Delta B=$27.41mT,旋磁比$\gamma = 1.831\times 10^5$T、$g$因子$g=1.655$和弛豫时间$\tau = 1.586\times 10^{-4}$s。